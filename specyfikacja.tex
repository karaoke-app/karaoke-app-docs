\documentclass[11pt]{article}

\usepackage[T1]{fontenc}
\usepackage[utf8]{inputenc}
\usepackage[pdfusetitle]{hyperref}
\usepackage{graphicx}
\usepackage{float}
\usepackage[MeX]{polski}
\usepackage{tabularx}
\usepackage[a4paper,left=3cm,right=3cm,top=2.5cm,bottom=2.5cm,includeheadfoot]{geometry}
\usepackage{titlesec}
\usepackage{indentfirst}
\usepackage{hyperref}
\usepackage[normalem]{ulem}

\title{Specyfikacja karaoke-app}
\author{Grupa F}

\begin{document}
  \maketitle

  \section{Wprowadzenie}
  Aplikacja \textit{karaoke-app} ma za zadanie udostępnienie popularnej rozrywki, jaką jest Karaoke, poprzez wygodny i intuicyjny interfejs dostępny z poziomu przeglądarki internetowej. Użytkownik wybiera utwór spośród dostępnych w katalogu, lub dodaje nowy poprzez wizualny kreator dostępny dla zalogowanych użytkowników. Podczas odtwarzania utworu wyświetlany jest tekst z wyróżnioną aktualną linią do zaśpiewania (według czasu).

  \section{Zespół}
  \begin{itemize}
    \item Dawid Bury (tester)   
    \item Daniel Hajduk (frontend developer)
    \item Adrian Korczykowski (dev-ops)
    \item Łukasz Poterała (backend developer)
    \item \underline{Mariusz Zdancewicz} (PM/frontend developer)
  \end{itemize}

  \section{Funkcjonalności}
  Aplikacja powinna implementować następujące funkcjonalności:
  \begin{itemize}
    \item uwierzytelnianie użytkowników (także za pomocą wybranych social media),
    \item katalog utworów:
    \begin{itemize}
      \item podział na kategorie,
      \item wyszukiwarka utworów,
      \item filtrowanie/sortowanie;
    \end{itemize}
    \item moduł karaoke:
    \begin{itemize}
      \item integracja z serwisem wideo,
      \item wyświetlanie tekstu utworu,
      \item wyróżnianie aktualnej linii tekstu (według czasu),
      \item sterowanie;
    \end{itemize}
    \item profil użytkownika:
    \begin{itemize}
      \item wyświetlanie podstawowych informacji o użytkowniku,
      \item ulubione utwory,
      \item utworzone playlisty,
      \item ustawienia prywatności;
    \end{itemize}
    \item wizualny kreator utworu:
    \begin{itemize}
      \item wybór ścieżki utworu,
      \item wprowadzanie metadanych utworu,
      \item wprowadzanie tekstu utworu (z własnego źródła),
      \item możliwość ustawienia znaczników czasu dla poszczególnych linii tekstu,
      \item przypisiwanie utworu do kategorii,
      \item dostęp tylko dla zalogowanych użytkowników;
    \end{itemize}
    \item playlisty:
    \begin{itemize}
      \item możliwość utworzenia przez użytkownika,
      \item automatyczne przełączanie utworów,
      \item ustawienia prywatności,
      \item losowanie utworów na playliście;
    \end{itemize}
    \item panel administratora:
    \begin{itemize}
      \item zarządzanie użytkownikami,
      \item zarządzanie katalogiem utworów,
      \item moderacja utworów utworzonych za pomocą kreatora;
    \end{itemize}
    \item ocena utworów,
    \item zgłaszanie problemów z utworem,
    \item analiza ruchu.
  \end{itemize}
\end{document}